\documentclass{article}
\usepackage{graphicx} % Required for inserting images
\usepackage[utf8]{inputenc}  % usar símbolos y acentos desde el teclado
\usepackage{amsmath,amssymb}
\usepackage[spanish]{babel}
\usepackage{booktabs}
\usepackage[table]{xcolor}
\usepackage{rotating}
\usepackage{wrapfig} % Figuras al lado del texto
\usepackage[rflt]{floatflt} % texto se acomode al contorno de la figura


\title{Sesión 007 de LaTeX}
\author{Jorge Luis Mírez Tarrillo}
\date{Agosto 28, 2024}

\begin{document}

\maketitle

\section*{Sesión 28 de agosto del 2024}

Tanto \cite{Goosens} y \cite{Lamport} y otros ejemplos son del uso de la bibliograf\'ia.

\section{Citas bibliogr\'aficas}

\begin{thebibliography}{99}
\bibitem[1]{Goosens} M. Goossens; F. Mittelbach; A. Samarin.
                {\it The \LaTeX Companion}. Addison-Wesley. 1993.
\bibitem[2]{Lamport} L. Lamport. {\it \LaTeX}. Addison-Wesley. 1996.
\end{thebibliography}

\section{Conclusiones}
El archivo *.tex estar\'a disponible en el UNIVirtual de OTI UNI luego de la sesión de hoy.

\end{document}