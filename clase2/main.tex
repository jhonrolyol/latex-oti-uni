\documentclass{article}
\usepackage{graphicx} % Required for inserting images
\usepackage{multicol}

\title{Mi primer archivo hecho en LaTeX}
\author{Jorge Luis Mírez Tarrillo}
\date{Agosto 07, 2024}

\begin{document}

\maketitle

\section*{Resumen}
Este es un primer documento que estoy haciendo como parte de mi aprendizaje en LaTeX.

Es organizado por la OTI de la Universidad Nacional de Ingeniería, en Lima, Perú.

Dirigido a todos los estudiantes y egresados del país más bello del mundo: el \textbf{Perú} \footnote{en donde se encuentra la ciudadela de MacchuPicchu}. \textit{Sean todos bienvenidos}.

\begin{multicols}{2}
    En la Universidad Nacional de Ingeniería, usualmente se organizan Cursos Gratuitos para la Comunidad UNI y también para todos los estudiantes y egresados de las universidades del Perú. Se les invita a estar pendiente de las página web y las redes sociales de la Universidad Nacional de Ingeniería.
\end{multicols}

\textbf{\textit{Sean todos bienvenidos}}

% ejemplo de pie de página
% \footnote{Ejemplo de pie de página}

\section*{Introducci\'on}

\section{Introducci\'on}
Los países limítrofes del Perú son:
\begin{enumerate}
    \item Ecuador.
    \item Colombia.
    \item Bolivia.
    \item Brasil.
    \item Chile.
\end{enumerate}

Los países limítrofes del Perú son:
\begin{itemize}
    \item Ecuador.
    \item Colombia.
    \item Bolivia.
    \item Brasil.
    \item Chile.
\end{itemize}

\begin{center}
    \textbf{\textit{PERU}}
\end{center}

\section{Marco Te\'orico}

\subsection{Ecuaciones}
\begin{equation}
    \alpha + \beta = \gamma
\end{equation}

\subsection{Definición de Conceptos}
En la presente tesis se usan los siguientes conceptos:
\begin{itemize}
    \item \textbf{Temperatura:} Representa el movimiento térmico de la materia.
    \item \textbf{Densidad:} Nos indica el grado de cohesión de la materia.
\end{itemize}

\subsubsection{Efectos de la presión sobre una tubería}
El golpe de ariete es muy perjudicial en los sistemas de tubería.

\subsubsection{Efectos de la temperatura sobre una tubería}
Una excesiva temperatura puede deformar permanentemente la tubería.

Por lo tanto, \underline{Todos somos Perú}

\section{Resultados}

\section{Discusión}

\section{Conclusiones}

\section{Recomendaciones}

\end{document}
