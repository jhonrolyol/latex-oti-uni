\documentclass{article}
\usepackage{graphicx} % Required for inserting images
\usepackage{multicol}
\usepackage{amsmath,amssymb} 
\usepackage{booktabs}
\usepackage[table]{xcolor}


\title{Sesión 005 de LaTeX}
\author{Jorge Luis Mírez Tarrillo}
\date{Agosto 20, 2024}

\begin{document}

\maketitle

\section*{Sesión 20 de agosto del 2024}

Vamos a realizar una primera tabla en la presente sesión. \\

Ejemplo 1. El uso del comando tabular.\\

\begin{tabular}{|c|c|c|} \hline  
    $p$ & $q$ & $p \rightarrow q$  \\ \hline
    0 & 0 & 1  \\ 
    0 & 1 & 1  \\ \cline{1-2}
    1 & 0 & 0  \\
    1 & 1 & 1  \\ \hline
\end{tabular} 
\newline

Ejemplo 2. Para lo que es una tabla flotante se usa el comando table (ver Tabla \ref{tab:01}).

\begin{table}[h]
    \centering
    \begin{tabular}{|c|c|} \hline
        1 & 2 \\ \hline
        3 & 4 \\ \hline
    \end{tabular}
    \caption{\textbf{Tabla ejemplo nro 2}}
    \label{tab:01}
\end{table}

Ejemplo 3. Ejemplo de tabla con nombre en cada columna (ver Tabla \ref{tab:02}).

\begin{table}[h]
    \centering
    \begin{tabular}{|c|c|c|} \hline
        \textbf{Nro.} & \textbf{Peso [kg]} & \textbf{Talla [cm]} \\ \hline
        1 & 40 & 150 \\ \hline
        2 & 50 & 155 \\ \hline
        3 & 46 & 140 \\ \hline
        4 & 48 & 148 \\ \hline
    \end{tabular}
    \caption{\textbf{Tabla ejemplo nro 2}}
    \label{tab:02}
\end{table}

Ejemplo 4. Ejemplo de tabla tipo IEEE (ver Tabla \ref{tab:03}).

\begin{table}[h]
    \centering
    \begin{tabular}{c c c} \hline
        \textbf{Nro.} & \textbf{Peso [kg]} & \textbf{Talla [cm]} \\ \hline
        1 & 40 & 150 \\ \hline
        2 & 50 & 155 \\ \hline
        3 & 46 & 140 \\ \hline
        4 & 48 & 148 \\ \hline
    \end{tabular}
    \caption{\textbf{Tabla ejemplo tipo recomendación de IEEE}}
    \label{tab:03}
\end{table}

Ejemplo 5: Usando otra vez table con tres filas y tres columnas

\begin{table}[]
    \centering
    \begin{tabular}{|c|c|c|} \hline
        0 & 0 & 1  \\ 
        0 & 1 & 1  \\ \cline{1-2}
        1 & 0 & 0  \\ \hline
    \end{tabular}
    \caption{Caption}
    \label{tab:my_label}
\end{table}

Ejemplo 6: Usando color en tablas.

\begin{table}[h!]
    \centering
    \rowcolors{1}{}{gray!20}
    \begin{tabular}{c|c}
       \rowcolor{yellow} $x_{n+1}$  & $x_{n+1} - x_{n}$ \\
         1 & 2 \\ 
    \end{tabular}
    \caption{Iteración de Newton para $x^2-cos(x)-1=8$}
    \label{tab:my_label}
\end{table}

\end{document}